\documentclass[12pt]{article}
\usepackage[utf8]{inputenc}

\title{\textbf{Farmer merchant auction system to automate marketing of agricultural goods through ecommerce}}
\author{Bsclmr151920}
\date{25 June 2022}
\usepackage{url}
\usepackage{setspace}
\onehalfspacing
\usepackage{graphicx}
\graphicspath{{./images/} }
\begin{document}
\maketitle
\pagestyle{plain}
\pagebreak
\begin{abstract}
    Technological advancements of the internet have lead to creation of millions of ecommerse systems that have facilitated
    easier selling of goods and services just by a click of a button. Unfortunately,it has not been  absorbed well in the farming
    sector due to bias opinion's that \emph{farming is not profitable}. This has discouraged investors focus on farming due to this prejudice causing farmers
    to suffer economically and be exploited by middlemen. With the economic opportunity farming offers,it makes sense to develop a platform that connects merchants and farmers directly to enable
    a consistent market of goods produced by the farmer. The following project discusses on a proposed  mobile merchant-farmer auction system that utilizes ecommerse technologies for example
    \emph{price tooling,shipping and tracking information,smart search and payment integration systems} with a core aim to facilitate easier ecommerse services on the farming sector.
\end{abstract}
\pagebreak
\tableofcontents
\pagebreak
\section{Introduction}
\subsection{Background}
Agriculture,the reason humans have civilizations started in the fertile crescent,in western
asia in the regions of mesapotamia and levant and some other parts of the world including
norther china and central america around 11500 years ago during the neorithic era(New stone age)\cite{r1}
. The advent of agriculture made it possible for the human population to increase many times more than what could be supported by gathering and hunting.  Wheet,peas,lentils,chickpeas and flax were the first crops to be planted by the neorithic ancestors of the human species. Goats,sheep,reindeer and cattle were also domesticated and became more reliable than  wild game animals which humans had relied on as the source of meat for thousands of years.
Evidence suggest that humans invented irrigation around forth millennium
BC in mesapotamia and egypt to protect their crops from drought . The oldest method
of irrigation made use of \textit{qanats},man-made underground
streams which are actually being used today in some parts of Middle east.
The bronze age saw people invent metal work which made creation of stronger farm tools possible.
Ancient civilizations had  complex agriculture systems. The roman civilization \cite{r16} used crop rotation in
the middle ages(5th century) and had already invented the manorial economic system(serfdom) which facilitated
agriculture commerce where crops were exported to neighboring kingdoms. It is intresting to note
that different provinces in the roman kingdom  specialized in the production of diffrent crops depending
on the soil type of their area. Provinces like the \textit{po valley}(now northern italy ) specialized in cereal production,\textit{Etruria} specialized
in wheat production and \textit{campania} produced wine. Agriculture was soo important to the romans that any person who
destroyed his/her neighbors crops received the death penalty. The chinese also practiced agriculture extensively.
The Qin and Han Dynasty had a nationwide granary systems and practiced sericulture(silk production) extensively which facilitated to the creation of the silk road(a 6400km road that connected china with rome). The chinese
had already invented hydraulic systems for example the hydraulic trip hammer used for pounding grain to produce flour and the square-pallet chain pump  before the first century showing that agriculture
facilitated a lot in the technological area.
Today,agriculture is very advanced as compared to the old ages.
Systems like GPS-based agriculture,satellite-derived data,pervasive automation, RFID Technology and Minichromosomal technology are being used 
to increase productivity,reduce workload and costs and utilize the available resources in a higher percentage. 
Like many other parts of the word, agriculture has always been the key source of kenyan's GDP since independence. In 2006 more than 75\%
of Kenyans relied on farming as their primary source of employment and many do soo even today\cite{r3}.
Kenya is one of the leading exporter of coffee,tea and fresh produce for example cabbage and mango’s.
It is also a leader in pyrethrum production,United states being the main consumer of this produce in Kenya's list of consumers.\cite{r4}. The agricultural sector contributed approximately 1,409 billion Kenyan shillings (KSh) to
Kenya's Gross Domestic Product (GDP) in the first half of 2021. (around 12.5 billion U.S. dollars)\cite{r5}
\subsection{Target Group}
The proposed system has two target groups:
\begin{itemize}
    \item The Farmer who wish to sell their product online to reliable customers.
    \item The Merchant who wish to buy farm produce at affordable rates without relying on middlemen
\end{itemize}

\subsection{Problem statement}
Marketing of Agricultural goods(especially perishable goods) has always been a challenge. Research suggest that farmers
in Kenya loose about 95 million litres of milk each year accounting to US\$22.5 million dollars. Developed countries loose
even more money with United states loosing US\$48.3  billion,Australia loosing US\$10.5 billion and China loosing 50 million tonnes of grain which would
account for billions of dollars\cite{r6}.
In total the world looses about US\$1 trillion. Although  wastage of agricultural products can be attributed to diffrent challanges,lack of market is a major cause.
Farmers should not struggle finding market for their produce,soo do merchants finding supply.There are cases where farmers plant
alot of plants,use their resources to take care of the crops but end up feeding the produce to their livestock because there is no market. Merchant on the other hand buy farm
produce from middlemen who exploit them by doubling the price of farm produce.
\subsection{Scope}
The scope of the proposal is to come up with a project that enables farmers market their produce online and expect customers within a defined time period. The project
will also enable merchants buy farm produce from farmers at agreed prices. The project is intended to serve the Kenya population in all counties
but with high priority in urbanized areas and agricultural counties in Kenya.
\subsection{General Objectives}
The aim of the proposal is to come up with a project that will ensure farmers market their produce
online with ease creating a stable supply of goods for merchants
\subsection{Specific Objectives}
The proposed project will contain three panels:
\begin{itemize}
    \item The farmer panel.
    \item The merchant panel
    \item The transportation panel

\end{itemize}
The farmer panel will allow farmers create an account,post their products in the project stating the quantity,
their location and the price of the produce. The merchant panel will allow merchants create an account and bid
the products. The project will also contain a transportation panel for both merchants and farmers  request transportation services.
After a successful bidding,the merchant will be able to communicate with the farmer directly and the farmer may become the supplier
of the said merchant and may decide not to post his/her produce on the dashboard but rather,deal with the merchant directly in the project.
The revenue will be generated using three business models:
\begin{itemize}
    \item \textbf{Transaction fee revenue mode} - each time a merchant wants to bid a product,
          he/she has to include additional transaction costs depending on the price of the product. Once a transaction is complete
          ,10\% of the returns to the farmer will be paid to the business  as service fees.
          This way the business(auction project) will get money from both parties. This would maximize the profit.
    \item \textbf{Affiliate revenue model } - Merchants who have their own websites can pay affiliate costs to the auction project so as to link and advertise their products.
          They shall pay the affilliate costs depending on the number of customers who are redirected to their site.
    \item \textbf{Sales revenue model} - as stated above,the business  will offer transportation services using the transportation panel at affortable rates. This way both merchants and farmers
          can easily transport their produce.
\end{itemize}
\subsection{Justification}
The system will enable farmers find potential customers much easily and faster hence creating a consistent market of their produce
and solve farm produce wastage that occur due to many issues one of them being lack of market.
The system will also create  a consistent supply of farm produce to merchants,hence feeding the high demand
of farm produce in Kenya. Because the system will be a B2B ecommerce system,time will be saved as everthing will operate virtually.
According to the international fund of agricultural development,80\% of  the rural
population of Kenya rely on the agriculture sector as their primary source of employment.
This means that more than 20 million Kenyans are small/medium scale farmers\cite{r7}. This shows that in the business perspective,the system may attact high
number of potential customers and solve the problems that they may be facing.


\subsection{Limitations}
The main challange that may be faced in the system is cyber attack. A hacker may decide
to compromise the account of  customers and steal their credentials. He/she may use those credentials
to gain access to customer  online payment system accounts potentially stealing their income. To solve
this issue,the system will use two factor authentication meaning that if the credentials are stolen,the hacker shall 
not be able to access the customers account as two factor authentication provides additional layer of protection. 
The system will also use firewalls and authentication middleware to protect itself from common hacking schemes.
\pagebreak
\section{Literature Review}
The goal of this literature review is to review e-commerce in agriculture, the problems facing agriculture in the e-commerce field,
the potential of agriculture e-commerce and how the problems can be resolved.


\subsection{Theoretical Review}
\subsubsection{Ecommerse in Agriculture}
Mueller \cite{r12}  defines e-commerce  as a business transactions conducted over the internet.
This transactions may include services or material goods. Mueller further states that
e-commerce transaction's are classified according to partner involved namely customers,business and government .
With this types of partners listed six business combinations are possible.
For this research we will focus on just one B2B(Business to Business) model.
In the last two decades,e-commerce has evolved significantly and its relevance increased.
GSMA(Global System for Mobile Communications) \cite{r13} argues that global e-commerce retail sales are now valued at US\$3 trillion accounting to 11.9\% of all global sales in 2018.
This means that many people have adopted the internet and shifted their preference to e-commerce.
Mobile platform adoption has partly facilitated to this growth. GSMA emphasize that in 2018,e-commerce transactions facilitated by mobile money grew
79\% in value with companies like jumia in kenya,payTM in india and Ozon in Russia offering  some form of mobile digital payment solution.
\begin{center}
      \includegraphics[width=15cm,height=10cm]{mobilepayment.png}
\end{center}

This shows that e-commerce has a large market base in many countries as millions of people worldwide have switched to the internet hence it will be utilised by thousands of farmers and merchants who have access to the internet.
Many features have facilitated to e-commerce one major feature being  Cloud computing. By definition,it is the provision of computer or IT infrastructure
through the internet\cite{r8}. Scientists and researchers all over the world are proposing cloud computing to be used in agricultural processes in order to increase
production and ensure optimal distribution of agricultural products. It is being encouraged because of the following reasons:
\begin{enumerate}

      \item \textbf{Automation of land records: }Hori,Yamazaki and Kawashima \cite{r10} argue that cloud computing storage facilities store land records with descriptions related to that specific land,
            such as soil analysis results and production history, among other things.
            Digital records as opposed to physical records which are difficult to analyses,store or manipulate are stored in this cloud storage facilities hence facilitating automation.
            This records are used to make critical decisions for example determine fertility of the land,
            the amount of crops the land can support and the soil type which aid in crop selection.
      \item \textbf{Access to IT resources at a low cost: } According to 	Hori,Yamazaki and Kawashima \cite{r10} ,cloud computing enables low-cost access to massive amounts of IT resources.It operates on
            a pay-per-use model where farmers don't need to own IT resources,but rather rent them from the cloud .
            This is the most cost effective and dependable method of obtaining resources
      \item \textbf{Weather forcasting}  According to Goraya, Singh and Kaur \cite{r11} cloud computing facilitates weather forecasting and  storage of weather forecasting  data  which enables farmers make crop selection decisions.
            This is especially useful in africa where farmers rely on rain water to grow crops
      \item \textbf{Storage of large amounts of data: }Cloud computing provides large capacity data stores for storing large amounts of data and information.
            Agriculte related data for example  crop information, weather information, market information, farmer experiences with agricultural processes, pesticide information and prescriptions can be easily stored on the cloud \cite{r9}.
            This data can then easily be accessed by the farmer.
            Data on the market conditions of various crops is useful in making crop selection decisions hence avoiding losses attributed to poor decision.
\end{enumerate}
With cloud computing,farmers can answer critical questions for example the market opportunity of a particular crop,how much time the crops will take in order to be harvested,will the weather
support the growth of a particular crop,how much will the crop cost in the e-commerce market and many more critical questions. This facilitates e-commerse as the farmer knows what ought to be done and will not invest
in practices that may potentially be unprofitable for him/her.
Cornelisse \cite{r17} argues that promotion of agricultural goods,packaging and distribution,product compatibility
where some products may not be appropriate for e-commerce sale,merchant product uncertainty,technical management
and support will be the key challenges that  will affect the agricultural e-commerce.

\subsection{Similar projects}


\subsubsection{Jiji}
Jiji is an e-commerce website built to auction merchant products to customers.
If a user likes a product auctioned on Jiji,
He/she can call the merchant through the contact provided in the website. The user
pays for a product once he/she has met the seller in the product store or a safe public
place and inspected the product he requested. The user can also request delivery and pay once the
product has been delivered.

\textbf{Selling on jiji: }Register using email and phone number or use OAuth(for example using facebook
or google) to register.take pictures of the items you want to sell. Enter a fair price,select attributes and
send your advert to review. If everything is ok,Jiji will auction sellers product in their website and send a notification
one the product is live.

\textbf{Buying on jiji: }Search for an item to buy using the search panel and filters. Contact the seller when ready
to buy using jiji chat or phone contact listed below selected product. Order for delivery from seller or pick
product from seller shop. Pay if its the product you ordered.
\pagebreak
\begin{center}
    \includegraphics[width=10cm,height=10cm]{jiji.png}
\end{center}
\subsubsection{Mfarm}
Mfarm is an agriculture e-commerce website that enables farmers know up-to-date market prices of
farm products via app,website or sms. It also connects farmers with buyers directly,cutting out the middlemen\cite{r14}.
Mfarm also developed a group selling tool to allow small scale farmers team up hence selling their
produce in bulk to merchants which will fetch more money to them. Mfarm also has a group buying tool that allows
farmers pool resources to negotiate better prices for example fertilizer. Transactions are all handled by Mfarm integrated mobile money transfer system(That uses Mpesa).
When a Merchant places an order through MFarm, the farmer delivers his or her produce to the designated collection point and sends a message to confirm delivery.
The buyer then collects the produce and sends a message to MFarm to verify the quantity and quality.
When that is confirmed and the order is fulfilled, MFarm transfers the funds to the farmer's account.
When placing larger orders involving multiple farmers, the money is divided among farmers accounts

\begin{center}
    \includegraphics[width=15cm,height=17cm]{mfarm.png}
\end{center}
\pagebreak

\subsubsection{Twiga foods}
Twiga foods works almost similar to Mfarm listed above. It connects farmers and small shopkeepers
via mobile app. If a vendor runs out of onions, for example, they can place an order on the Twiga app,
and the company will deliver them within 24 hours, either from its warehouse or directly from a smallholder farmer.
The products sold by Twiga are much cheaper than those sold by middlemen who inflate prices.
\begin{center}
    \includegraphics[width=15cm,height=17cm]{twiga.png}
\end{center}
\subsection{Conceptual Framework}
The diagram below which contains the inputs,processes and outputs is the conceptual framework.
The user registers to the system using his/her name,email,phone number and business types. This
data is validated and is stored in the database if it does not exist. The system then prompts the user to login
using the created account.If the details the user enters don't exist in the database,the user is redirected to
the register page.  If the user logs in and the details exist,the system checks the user type(is the user a merchant or farmer).
If the user is a merchant,he/she is prompted to fill in business details for example business name,its registration details and his/her payment account.
Once the details have been saved to the database,the Merchant can now check auction products and order any of them.
If the user is a farmer,he/she is requested to fill in the product type intended to be auctioned and his/her payment account. Once saved to 
the database,the farmer can view order listings and auction products


\begin{center}
    \includegraphics[width=16cm,height=20cm]{conceptualFramework.png}
\end{center}
\pagebreak


\section{Methodology}
The methodology below will explain what i did in my research and how i did it so as to
show the validity and reliability of my research.
\subsection{Data Source}
The research problems that were being investigating were the challenges farmers face when adopting e-commerce in agriculture and the problems farmers and merchants face when marketing and buying agricultural goods respectively.
Strategies to solve this problems were also investigated. The research used a survey method (for example a survey from Timalsina \cite{r15}) and a case study.
The primary source of data was merchants and farmers and the secondary was surveys,reports,inbooks and research done on the topics specified.
Quantitative methods were also used in may cases in the research.

\subsection{Data Collection}
\subsubsection{Questionnaire}
Questionnaires,which had a likert scale were used to elicit required data. These questionnaires
included questions pertaining to challenges farmers and merchants face in
e-commerce,challenges both farmer's and merchants face during marketing of agricultural goods and strategies to solve this challenges

\subsubsection{Observations of current trends}
Current trends in agriculture were also researched on.
This data was mainly  referenced from blogs and youtube videos that explained in
details the current trends in agriculture especially in the context of Iot(internet of things)
and e-commerce. Research articles and government websites were also used as a reference.
\subsubsection{Existing data}
Many sources were from research articles and thesis from different researchers.
Authentic websites for example wikipedia and research companies for example
Global System for Mobile Communications and International fund for agricultural development were used.

\subsection{System design}
\begin{center}
    \includegraphics[width=16cm,height=20cm]{systemdesign.png}
\end{center}




\bibliographystyle{IEEEtran}
\bibliography{references}
\end{document}
