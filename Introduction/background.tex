\subsection{Background}
Agriculture,the reason humans have civilizations started in the fertile crescent,in western
asia in the regions of mesapotamia and levant and some other parts of the world including
norther china and central america around 11500 years ago during the neorithic era(New stone age)\cite{r1}
. The advent of agriculture made it possible for the human population to increase many times more than what could be supported by gathering and hunting.  Wheet,peas,lentils,chickpeas and flax were the first crops to be planted by the neorithic ancestors of the human species. Goats,sheep,reindeer and cattle were also domesticated and became more reliable than  wild game animals which humans had relied on as the source of meat for thousands of years.
Evidence suggest that humans invented irrigation around forth millennium
BC in mesapotamia and egypt to protect their crops from drought . The oldest method
of irrigation made use of \textit{qanats},man-made underground
streams which are actually being used today in some parts of Middle east.
The bronze age saw people invent metal work which made creation of stronger farm tools possible.
Ancient civilizations had  complex agriculture systems. The roman civilization \cite{r16} used crop rotation in
the middle ages(5th century) and had already invented the manorial economic system(serfdom) which facilitated
agriculture commerce where crops were exported to neighboring kingdoms. It is intresting to note
that different provinces in the roman kingdom  specialized in the production of diffrent crops depending
on the soil type of their area. Provinces like the \textit{po valley}(now northern italy ) specialized in cereal production,\textit{Etruria} specialized
in wheat production and \textit{campania} produced wine. Agriculture was soo important to the romans that any person who
destroyed his/her neighbors crops received the death penalty. The chinese also practiced agriculture extensively.
The Qin and Han Dynasty had a nationwide granary systems and practiced sericulture(silk production) extensively which facilitated to the creation of the silk road(a 6400km road that connected china with rome). The chinese
had already invented hydraulic systems for example the hydraulic trip hammer used for pounding grain to produce flour and the square-pallet chain pump  before the first century showing that agriculture
facilitated a lot in the technological area.
Today,agriculture is very advanced as compared to the old ages.
Systems like GPS-based agriculture,satellite-derived data,pervasive automation, RFID Technology and Minichromosomal technology are being used 
to increase productivity,reduce workload and costs and utilize the available resources in a higher percentage. 
Like many other parts of the word, agriculture has always been the key source of kenyan's GDP since independence. In 2006 more than 75\%
of Kenyans relied on farming as their primary source of employment and many do soo even today\cite{r3}.
Kenya is one of the leading exporter of coffee,tea and fresh produce for example cabbage and mango’s.
It is also a leader in pyrethrum production,United states being the main consumer of this produce in Kenya's list of consumers.\cite{r4}. The agricultural sector contributed approximately 1,409 billion Kenyan shillings (KSh) to
Kenya's Gross Domestic Product (GDP) in the first half of 2021. (around 12.5 billion U.S. dollars)\cite{r5}