\subsection{Theoretical Review}
\subsubsection{Ecommerse in Agriculture}
Mueller \cite{r12}  defines e-commerce  as a business transactions conducted over the internet.
This transactions may include services or material goods. Mueller further states that
e-commerce transaction's are classified according to partner involved namely customers,business and government .
With this types of partners listed six business combinations are possible.
For this research we will focus on just one B2B(Business to Business) model.
In the last two decades,e-commerce has evolved significantly and its relevance increased.
GSMA(Global System for Mobile Communications) \cite{r13} argues that global e-commerce retail sales are now valued at US\$3 trillion accounting to 11.9\% of all global sales in 2018.
This means that many people have adopted the internet and shifted their preference to e-commerce.
Mobile platform adoption has partly facilitated to this growth. GSMA emphasize that in 2018,e-commerce transactions facilitated by mobile money grew
79\% in value with companies like jumia in kenya,payTM in india and Ozon in Russia offering  some form of mobile digital payment solution.
\begin{center}
      \includegraphics[width=15cm,height=10cm]{mobilepayment.png}
\end{center}

This shows that e-commerce has a large market base in many countries as millions of people worldwide have switched to the internet hence it will be utilised by thousands of farmers and merchants who have access to the internet.
Many features have facilitated to e-commerce one major feature being  Cloud computing. By definition,it is the provision of computer or IT infrastructure
through the internet\cite{r8}. Scientists and researchers all over the world are proposing cloud computing to be used in agricultural processes in order to increase
production and ensure optimal distribution of agricultural products. It is being encouraged because of the following reasons:
\begin{enumerate}

      \item \textbf{Automation of land records: }Hori,Yamazaki and Kawashima \cite{r10} argue that cloud computing storage facilities store land records with descriptions related to that specific land,
            such as soil analysis results and production history, among other things.
            Digital records as opposed to physical records which are difficult to analyses,store or manipulate are stored in this cloud storage facilities hence facilitating automation.
            This records are used to make critical decisions for example determine fertility of the land,
            the amount of crops the land can support and the soil type which aid in crop selection.
      \item \textbf{Access to IT resources at a low cost: } According to 	Hori,Yamazaki and Kawashima \cite{r10} ,cloud computing enables low-cost access to massive amounts of IT resources.It operates on
            a pay-per-use model where farmers don't need to own IT resources,but rather rent them from the cloud .
            This is the most cost effective and dependable method of obtaining resources
      \item \textbf{Weather forcasting}  According to Goraya, Singh and Kaur \cite{r11} cloud computing facilitates weather forecasting and  storage of weather forecasting  data  which enables farmers make crop selection decisions.
            This is especially useful in africa where farmers rely on rain water to grow crops
      \item \textbf{Storage of large amounts of data: }Cloud computing provides large capacity data stores for storing large amounts of data and information.
            Agriculte related data for example  crop information, weather information, market information, farmer experiences with agricultural processes, pesticide information and prescriptions can be easily stored on the cloud \cite{r9}.
            This data can then easily be accessed by the farmer.
            Data on the market conditions of various crops is useful in making crop selection decisions hence avoiding losses attributed to poor decision.
\end{enumerate}
With cloud computing,farmers can answer critical questions for example the market opportunity of a particular crop,how much time the crops will take in order to be harvested,will the weather
support the growth of a particular crop,how much will the crop cost in the e-commerce market and many more critical questions. This facilitates e-commerse as the farmer knows what ought to be done and will not invest
in practices that may potentially be unprofitable for him/her.
Cornelisse \cite{r17} argues that promotion of agricultural goods,packaging and distribution,product compatibility
where some products may not be appropriate for e-commerce sale,merchant product uncertainty,technical management
and support will be the key challenges that  will affect the agricultural e-commerce.
