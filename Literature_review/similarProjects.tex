\subsection{Similar projects}


\subsubsection{Jiji}
Jiji is an e-commerce website built to auction merchant products to customers.
If a user likes a product auctioned on Jiji,
He/she can call the merchant through the contact provided in the website. The user
pays for a product once he/she has met the seller in the product store or a safe public
place and inspected the product he requested. The user can also request delivery and pay once the
product has been delivered.

\textbf{Selling on jiji: }Register using email and phone number or use OAuth(for example using facebook
or google) to register.take pictures of the items you want to sell. Enter a fair price,select attributes and
send your advert to review. If everything is ok,Jiji will auction sellers product in their website and send a notification
one the product is live.

\textbf{Buying on jiji: }Search for an item to buy using the search panel and filters. Contact the seller when ready
to buy using jiji chat or phone contact listed below selected product. Order for delivery from seller or pick
product from seller shop. Pay if its the product you ordered.
\pagebreak
\begin{center}
    \includegraphics[width=10cm,height=10cm]{jiji.png}
\end{center}
\subsubsection{Mfarm}
Mfarm is an agriculture e-commerce website that enables farmers know up-to-date market prices of
farm products via app,website or sms. It also connects farmers with buyers directly,cutting out the middlemen\cite{r14}.
Mfarm also developed a group selling tool to allow small scale farmers team up hence selling their
produce in bulk to merchants which will fetch more money to them. Mfarm also has a group buying tool that allows
farmers pool resources to negotiate better prices for example fertilizer. Transactions are all handled by Mfarm integrated mobile money transfer system(That uses Mpesa).
When a Merchant places an order through MFarm, the farmer delivers his or her produce to the designated collection point and sends a message to confirm delivery.
The buyer then collects the produce and sends a message to MFarm to verify the quantity and quality.
When that is confirmed and the order is fulfilled, MFarm transfers the funds to the farmer's account.
When placing larger orders involving multiple farmers, the money is divided among farmers accounts

\begin{center}
    \includegraphics[width=15cm,height=17cm]{mfarm.png}
\end{center}
\pagebreak

\subsubsection{Twiga foods}
Twiga foods works almost similar to Mfarm listed above. It connects farmers and small shopkeepers
via mobile app. If a vendor runs out of onions, for example, they can place an order on the Twiga app,
and the company will deliver them within 24 hours, either from its warehouse or directly from a smallholder farmer.
The products sold by Twiga are much cheaper than those sold by middlemen who inflate prices.
\begin{center}
    \includegraphics[width=15cm,height=17cm]{twiga.png}
\end{center}